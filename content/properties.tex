
\section*{\underline{Properties}}

\subsection*{Bra-Ket (Dirac) notation}
\begin{itemize}
    \item \underline{\textbf{Ket:}}
        \begin{itemize}
            \item \(\ketp + \ket{\varphi} = \ket{\psi + \varphi}\)
            \item \(a \ketp = \ket{a \psi}\)
            \item \(\ketp = \brap^\dagger\)
        \end{itemize}

        Righthand-side rules:
        \begin{itemize}
            \item \( \Braket{\psi \mid \varphi + \zeta} = \Braket{\psi \mid \varphi} + \Braket{\psi \mid \zeta}\)
            \item \(\Braket{\psi \mid a\varphi} = a \Braket{\psi \mid \varphi}\)
        \end{itemize}

        Complex conjugate:
        \begin{itemize}
            \item \(\Braket{\psi \mid \varphi}^* = \Braket{\varphi \mid \psi}\)
        \end{itemize}
        
        All left side rules can be build ontop of these (antilinearly)
    \item Riesz Representation Theorem results in: \(\bra{\psi} \ket{\varphi} = \Braket{\psi \mid \varphi} = \psi \cdot \varphi\)
\end{itemize}

\subsection*{Matrices}
\begin{itemize}
    \item \href{https://en.wikipedia.org/wiki/Normal_matrix}{\textbf{\underline{Normal matrix:}}}

    A is normal if it commutes with its conjugate transpose $A^\dagger$:
    \[A \; normal \Leftrightarrow A^\dagger A = A A^\dagger\]
    
    \item \href{https://en.wikipedia.org/wiki/Unitary_matrix}{\textbf{\underline{Unitary matrix:}}}
        
    U is unitary if its matrix inverse $U^-1$ equals its conjugate transpose $U^\dagger$:
    \[U \; unitary \Leftrightarrow U^\dagger U = U U^\dagger = I\]
    \underline{Properties}:
    For any unitary matrix U of finite size, the following hold:
    \begin{adjustwidth}{2em}{0pt}
        \setlength{\abovedisplayskip}{0pt}
        \begin{flalign*}
            - \abs{\det(U)} &= 1 & \\
            - \det(U) \ &\eqqcolon e^{i\theta}, \theta \in \mathbb{R} & 
        \end{flalign*}
    \end{adjustwidth}


    \item \href{https://en.wikipedia.org/wiki/Hermitian_matrix}{\textbf{\underline{Hermitian matrix:}}}
    
    H is Hermitian if it's a square matrix that is equal to its conjugate transpose $H^\dagger$:
    \[H \; hermitian \Leftrightarrow H^\dagger = H\]
 
    For any hermitian matrix H of finite size, the following hold:
    \begin{itemize}
        \item All eigenvalues are real.
        \item Eigenvectors corresponding to distinct eigenvalues are orthogonal.
        \item In real-valued matrices, “Hermitian” just means symmetric \((A = A^T)\)
    \end{itemize}

    \item \href{https://en.wikipedia.org/wiki/Pauli_matrices}{\textbf{\underline{Pauli matrices:}}} \\
    All Pauli matrices \({M \in \{X, Y, Z\} }\) also notated as \({\sigma_i \mid i \in [3]}\) satisfy these properties:
    \begin{itemize}[label=]
        \item \(\sigma_j^2 = I\)
        \item \(\sigma_j \sigma_k = -\sigma_k \sigma_j \; \; \forall j, k \in [3] \mid j \ne k\)
        \item \([\sigma_j, \sigma_k] = \sigma_j \sigma_k - \sigma_k \sigma_j = 2i \sigma_l \; \; \forall (j, k, l) \in \mathrm{Cyc}(1, 2, 3)\)
    \end{itemize}

    \item \href{https://en.wikipedia.org/wiki/Diagonal_matrices}{\textbf{\underline{Diagonal matrices:}}} \\
    All Diagonal matrices \(D \in R^{n \times n}\) satisfy these properties:
    \begin{itemize}[label=]
        \item \(e^{D \cdot x} = \colvec{e^{d_{11} \cdot x} & \hdots & 0 \\ \vdots & \ddots & \vdots \\ 0 & \hdots & e^{d_{nn} \cdot x}}\)
    \end{itemize}

    \item \href{https://en.wikipedia.org/wiki/Density_matrix}{\textbf{\underline{Density matrix/operator:}}} \\
    The density operator \(\rho = \sumX_i \textcolor{blue}{p_i} \ketbra[\psi_i]\) has these properties:
    \begin{itemize}[label=-]
        \item \(\rho \; hermitian : \rho^\dagger = (\ketbra)^\dagger = \ketbra = \rho\)
        \item \(\rho \; idempotent : \rho^2 = \ketp \BraKet{} \brap = \ketbra = \rho \; | \text{ if \textbf{pure} state}.\)
        \item \(\tr[\rho] = \sumX_i p_i = \sumX_i \lambda_i = 1\)
        \item May be rewritten as such: \(\rho = \frac{I + (\vec{r} \cdot \vec{\sigma})}{2}\)
        \item $\rho$ is positive semi-definite so all eigenvalues are positive.
        \item The 2 eigenvalues of $\rho$ can be calulcated as such: \(\frac{1 \pm \norm{\vec{r}}}{2}\)
    \end{itemize}
\end{itemize}
