
\section*{\underline{Properties}}

\begin{itemize}
    \item \href{https://en.wikipedia.org/wiki/Normal_matrix}{\textbf{\underline{Normal matrix:}}}

    A is normal if it commutes with its conjugate transpose $A^\dagger$:
    \[A \; normal \Leftrightarrow A^\dagger A = A A^\dagger\]
    
    \item \href{https://en.wikipedia.org/wiki/Unitary_matrix}{\textbf{\underline{Unitary matrix:}}}
        
    U is unitary if its matrix inverse $U^-1$ equals its conjugate transpose $U^\dagger$:
    \[U \; unitary \Leftrightarrow U^\dagger U = U U^\dagger = I\]

    % \underline{Properties}:
    % For any unitary matrix U of finite size, the following hold: \\
    % \begin{itemize}[label=]
    %     \item 
    % \end{itemize}

    \item \href{https://en.wikipedia.org/wiki/Hermitian_matrix}{\textbf{\underline{Hermitian matrix:}}}
    
    H is Hermitian if it's a square matrix that is equal to its conjugate transpose $H^\dagger$:
    \[H \; hermitian \Leftrightarrow H^\dagger = H\]
 
    For any hermitian matrix H of finite size, the following hold:
    \begin{itemize}
        \item All eigenvalues are real.
        \item Eigenvectors corresponding to distinct eigenvalues are orthogonal.
        \item In real-valued matrices, “Hermitian” just means symmetric \((A = A^T)\)
    \end{itemize}
\end{itemize}
