\section{Quantum operations}
(Nielsen and Chuang section 8.2)

\subsection{Motivation and overview}

In general: changes of quantum states affected by unitary time evolution or wavefunction collapse during measurement.
Quantum operations (also called \underline{"quantum channels"}) are a (mathematical) generalization 
and unification of these concepts. \\

\noindent
Abstractly: \\
\begin{center}
    \(\rho' = \mathcal{E}(\rho)\) \\ with \(\mathcal{E} \coloneqq\) quantum operation and \(\rho \coloneqq\) density matrix
\end{center}

\noindent
Special cases:
\begin{itemize}
    \item unitary time evolution: \(\quop = U \cdot \rho \cdot U^\dagger\)
    \item measurements, with measurement operations ${M_m}$:
            \begin{align*}
                &\text{recall that } p(m) = \tr[M_m^\dagger M_m \rho] \ \ | \; p(m) \coloneqq \text{"probability of outcome m"} \\
                &\text{State after obtaining result m:} \\
                & \ \ \ \ \ \textcolor{blue}{\rho_m} = \frac{M_m \rho M_m^\dagger}{\tr[M_m^\dagger M_m \rho]} = \frac{M_m \rho M_m^\dagger}{p(m)} \\
                &\text{Corresponding quantum channel (without normalisation):} \\
                & \ \ \ \ \ \; \quop = M_m \rho M_m^\dagger \\
                & \ \ \ \ \ p(m) = \tr[\quop] \text{ is the probability that outcome m occurs.}
            \end{align*}
\end{itemize}

\noindent
Consider the scenario of performing a measurement, but not recording the outcome: \\
\indent $\leadsto$ density matrix after this process is the weighted sum over all possible outcomes:

\[\quop = \sumX_m p(m) \cdot \textcolor{blue}{\rho_m} = \sumX_m M_m \rho M_m^\dagger \coloneqq "\text{operator-sum (Kraus) representation of } \mathcal{E}"\]

\noindent
Different (but equivalent) perspectives on quantum operations:
\begin{itemize}[label=-]
    \item system coupled to environment (Stinespring dilation)
    \item operator-sum (Kraus) representation
    \item physically motivated axioms
    \item Choi matrix representation
\end{itemize}


\subsection{Environments and quantum operations}

"Open" quantum systems can be regarded as interactions between a:
\begin{itemize}
    \item principal quantum system (initially in state $\rho$)
    \item environment (initially in state $\rho_{env}$)
\end{itemize}

\noindent
The principal system interacts with the environment, i.e. a time evolution of the overall system 
described by some unitary U:

\begin{figure}[H]
    \begin{subfigure}{0.5\textwidth}
        \hspace{2em} Circuit diagram: 
        \vspace*{10pt}

        \begin{quantikz}
            \lstick{$\rho$} && \gate[wires=2]{U} && \\
            \lstick{$\rho_{env}$} &&             && \\
        \end{quantikz}
    \end{subfigure}
    \hfill
    \begin{subfigure}{0.45\textwidth}
        \ \ mathematical representation:
        \[U \cdot (\rho \otimes \rho_{env}) \cdot U^\dagger\]
    \end{subfigure}
\end{figure}

\noindent
Example:
\begin{figure}[H]
    \begin{quantikz}
        \lstick{$\rho$}         && \ctrl{1} \gategroup[2,steps=1,style={inner sep=6pt}]{$U_{\text{c-{\sc not}}}$} && \rstick{$\quop$} \\
        \lstick{$\ketbra[0]$}   && \targ{}  && 
    \end{quantikz}
\end{figure}

\noindent
Represents \(\rho   = \colvec{\rho_{00} & \rho_{01} \\ \rho_{10} & \rho_{11}} 
                    = \rho_{00} \ketbra[0] + \rho_{01} \ket{0} \bra{1} + \rho_{10} \ket{1} \bra{0} \rho_{11} \ketbra[1]\),
then:
 
\begin{flalign*}
    &\ \ \ \ U_{\text{c-{\sc not}}} \cdot (\rho \otimes \textcolor{forest}{\ketbra[0]}) U_{\text{c-{\sc not}}}^\dagger \\
    &= U_{\text{c-{\sc not}}} \cdot (\rho_{00} \ketbra[0\textcolor{forest}{0}] + \rho_{01} \ket{0\textcolor{forest}{0}} \bra{1\textcolor{forest}{0}} 
                                    + \rho_{10} \ket{1\textcolor{forest}{0}} \bra{0\textcolor{forest}{0}} \rho_{11} \ketbra[1\textcolor{forest}{0}]) U_{\text{c-{\sc not}}}^\dagger \\
    &= \hspace{3em} \hspace{8pt} \rho_{00} \ketbra[00] + \rho_{01} \ket{00} \bra{1\textcolor{red}{1}} + \rho_{10} \ket{1\textcolor{red}{1}} \bra{00} \rho_{11} \ketbra[1\textcolor{red}{1}]
\end{flalign*}

\begin{flalign*}
    \tr_{env}[\hdots]   &= \hspace{17pt} \rho_{00} \ketbra[0] \BraKet[0]{} + \rho_{01} \ket{0} \bra{1} \braket{1 | 0} + \rho_{10} \ket{1} \bra{0} \braket{0 | 1} \rho_{11} \ketbra[1] \BraKet[1]{} \\
                        &= \hspace{17pt}\rho_{00} \ketbra[0] + \rho_{11} \ketbra[1] \\
                        &= \hspace{15pt} \colvec{\rho_{00} & 0 \\ 0 & \rho_{11}} \\
                        &= \hspace{17pt} P_0 \rho P_0 + P_1 \rho P_1 \text{ with } P_0 = \ketbra[0], \; P_1 = \ketbra[1]
\end{flalign*}
\begin{center}
    (off-diagonal entries of $\rho$ are set to zero)
\end{center}


\subsection{Operator-sum (Kraus) representation}

Let \({\ket{e_k}}\) be an orthonormal basis of the environment quantum system, assume w.l.o.g. \(\rho_{env} = \ketbra[e_0]\)
(see tutorial 12: if environment is a mixed state, then can equivalently work with a pure state in a larger environment) \\
\begin{align*}
    &\textcolor{gray}{(1): \sumX_k \ketbra[e_k] = I \text{ on environment}} \\
    &\textcolor{gray}{(2): \tr[A \ketbra[j]] = \BraKet[j]{A}}
\end{align*}
\begin{align*}
    \leadsto \quop  &= \tr_{env}[U(\rho \otimes \ketbra[e_0]) U^\dagger] \\
                    &\overset{(1)}{=} \sumX_k \tr_{env}[U (\rho \otimes \ketbra[e_0]) U^\dagger \cdot (I \otimes \ketbra[e_k])] \\
                    &\overset{(2)}{=} \sumX_k \BraKet[e_k]{U (\rho \otimes \textcolor{forest}{\ketbra[e_0]}) U^\dagger} \\
                    &= \sumX_k \textcolor{blue}{E_k} \rho \textcolor{blue}{E_k^\dagger}
\end{align*}

\noindent
with $\textcolor{blue}{E_k}$ a complex matrix with entries \((\textcolor{blue}{E_k})_{lm} = \braket{l, e_k | U | m, \textcolor{forest}{e_0}}\)

\noindent
The $E_k$'s are called \underline{operation elements} or \underline{Kraus operators} of $\mathcal{E}$:

\begin{figure}[H]
    \begin{subfigure}{0.5\textwidth}
        \[\quop = \sumX_k \textcolor{blue}{E_k} \rho \textcolor{forest}{E_k^\dagger}\]
    \end{subfigure}
    \hfill
    \begin{subfigure}{0.45\textwidth}
        \begin{quantikz}
                                                    &  \gate[2]{\textcolor{blue}{U}}    & \measure{\rho}                         & \gate[2]{\textcolor{forest}{U^\dagger}}  & \\
            \lstick{$\textcolor{blue}{\bra{e_k}}$}  &  \qw                              & \push{\textcolor{blue}{\ket{e_0}} \textcolor{forest}{\bra{e_0}}} &                                          & \rstick{\textcolor{forest}{$\bra{e_k}$}}
        \end{quantikz}
    \end{subfigure}
\end{figure}

\noindent
Revisit example:
\begin{figure}[H]
    \begin{quantikz}
        \lstick{$\rho$}         && \ctrl{1} \gategroup[2,steps=1,style={inner sep=6pt}]{$U_{\text{c-{\sc not}}}$} && \rstick{$\quop$} \\
        \lstick{$\ketbra[0]$}   && \targ{}  && 
    \end{quantikz}
\end{figure}

\noindent
\((E_0)_{lm} = \braket{l, 0 | U_{CNOT} | m, 0}\) \hspace{3em} \(\leadsto E_0 = \colvec{1 & 0 \\ 0 & 0} = P_0\) \\
\((E_1)_{lm} = \braket{l, 1 | U_{CNOT} | m, 0}\) \hspace{3em} \(\leadsto E_1 = \colvec{0 & 0 \\ 0 & 1} = P_1\) \\
\textcolor{gray}{Look at official lecture for extra notes} $P_0$ and $P_1$ agree with previous derivation. \\

\newpage \noindent
If $\mathcal{E}$ is trace-preserving, then for any density matrix of $\rho$: \\
\[1 \overset{!}{=} \tr[\quop] = \tr[\sumX_k E_k \rho E_k^\dagger] = \sumX_k \tr[E_k \rho E_k^\dagger] \overset{*}{=} \sumX_k \tr[E_k^\dagger E_k \rho] = \tr[(\sum_k E_k^\dagger E_k) \rho]\]
\textcolor{gray}{*: cyclic invariance} \\
should hold for arbitrary $\rho$ with $\tr[\rho] = 1 \leadsto \textcolor{blue}{\sumX_k E_k^\dagger E_k = I}$ \\

\noindent
We allow for quantum channels with $\sumX_k E_k^\dagger E_k \textcolor{blue}{\le} I$ \\
\("A \le B"\) if B-A is positive semidefinite (p.s.d.) 
\begin{itemize}[label=]
    \item \(A \in \mathbb{C}^{n \times n} \) is called \underline{positive semidefinite} if A is Hermitian and \(\BraKet[V]{A} \ge 0\) for all \(v \in \mathbb{C}^n \Leftrightarrow\) all eigenvalue of A are $\ge$ 0
\end{itemize}

\noindent
\(\sumX_k E_k^\dagger E_k \overset{!}{\le} I\) stems from requiremet that $\tr[\quop] \overset{!}{\le} 1$

\noindent
Remark: \underline{Physical interpretation} of a quantum channel as measurement performed on environment with respect to ${e_k}$ basis, i.e. the measurement operators $\ketbra[e_k]$: \\

\noindent
For outcome k, state of principal system is: 
\[\rho_k \propto \tr_{env}[ \; \ketbra[e_k] U (\rho \otimes \ketbra[e_0]) U^\dagger \ketbra[e_k] \; ] = E_k \rho E_k^\dagger\],
i.e. $E_k$'s play the role of the measurement operators on the principal system.

\noindent
\underline{System-environment model of Krans representation} \\
Given trace-preserving \(\quop = \sumX_{k = 0}^{n-1} E_k \rho E_k^\dagger\), is there a corresponding system-environment representation? \\
Yes: Define model environment as vector space of dimension \textcolor{orange}{n} with orthonormal basis \({\ket{e_k}}_{k=0, \hdots, n-1}\), assume that environment starts in $\ket{e_0}$ state. \\

\noindent
Define unitary U via \(U \cdot \ketp \ket{e_0} \overset{\textcolor{red}{(*)}}{=} \sumX_k E_k \ketp \ket{e_k}\) and use \textcolor{blue}{matrix extension} 
to a unitary operator on the combined system. \\

\noindent
This is possible since for any principal quantum states \ketp, 
\begin{align*}
    \bra{\psi} \bra{e_0} U^\dagger U \ket{\varphi} \ket{e_0} &\overset{\textcolor{red}{(*)}}{=} \sumX_{k, k'} \braket{\psi | E_k^\dagger E_{k'} | \varphi} \braket{e_k | e_{k'}} \\
    &= \braket{\psi | \sumX_k E_k E_k^\dagger | \varphi} = \braket{\psi | \varphi} = I 
\end{align*} (completeness relation)

\noindent
U has the desired property since:
\begin{align*} 
    \tr_{env}[U (\rho \otimes \ketbra[e_0]) U^\dagger ] &\overset{\textcolor{red}{(*)}}{=} \sumX_{k, k'} \tr_{env}[ \; (E_k \rho E_{k'}) \otimes \ket{e_k} \bra{e_{k'}} \;] \\
    &= \sumX_{k, k'} E_k \rho E_{k'}^\dagger \underbrace{\braket{e_{k'} | e_{k}}}_{\delta_{kk'}} \\
    &= \sumX_k E_k \rho E_k^\dagger
\end{align*}

\subsection{6.4 Axiomatic approach to quantum operations}
Alternative viewpoint: physically motivated axioms which a quantum operation $\mathcal{E}$ has to obey:

\begin{itemize}[label=]
    \item A1: \(\tr[\quop]\) is the probability that the process $\mathcal{E}$ occurs, thus \(0 \le \tr[\quop] \le 1\) for all density matrices $\rho$.
    \item A2: \(\mathcal{E}\) is convex-linear: \\ \[\mathcal{E}(\sumX_i p_i \rho_i) = \sumX_i p_i \mathcal{E}(\rho_i) \text{ for any probability vector p and density matrices } \rho \]
    \item A3: \(\mathcal{E}\) is a \underline{completely positive} map: \\
                \begin{quote}
                $\mathcal{E}(A)$ must be positive semidefinite (p.s.d.) for any p.s.d. matrix A \\
                Moreover, when enlarging the principal system Q by another quantum system R, then \((I \otimes \mathcal{E})(A)\) must be p.s.d. for any p.s.d. A on the conbined system RQ.
                \textcolor{gray}{I on R and $\mathcal{E}$ on Q}
                \end{quote}

\end{itemize}

\subsection{Currently in Work}
\subsection{Currently in Work}

\subsection{Examples of quantum channels}
(Nielsen and Chuang section 8.3) \\

\noindent
Bit flip channel: flip \(\ketz \leftrightarrow \keto\) with probability \(1 - p, p \in [0,1]\)
\begin{flalign*}
    &E_0 = \sqrt{p} \cdot I, E_1 = \sqrt{1 - p} \cdot X \\
    &\mathcal{E}_{BF}(\rho) = \sumX_{k = 0}^1 E_k \rho E_k^\dagger = p \cdot \rho + (1 - p) X \rho X
\end{flalign*}

\noindent
Recall Block vector $\vec{r}$ of density matrix: \(\rho = \frac{I + \vec{r} \cdot \vec{\sigma}}{2}\)
\[\tr[\rho^2] = \frac{1}{2} (1 + \norm{\vec{r}}^2), \text{ see Exercise 11.2}\]
\begin{align*}
    &X \cdot X \cdot X = X \\
    &Y \cdot Y \cdot Y \; = \textcolor{orange}{-}Y \\
    &Z \cdot Z \cdot Z \; \, = \textcolor{orange}{-}Z
\end{align*}
\[\mathcal{E}_{BF}(\rho) = \frac{I + \vec{r'} \cdot \vec{\sigma}}{2} \text{ with } 
\vec{r'} = \colvec{p \ r_1 + (1 - p)r_1 \\ p \  r_2 + (1 - p)r_2 \\ p \ r_3 + (1 - p)r_3} 
= \colvec{r_1 \\ (2p - 1) r_2 \\ (2p - 1) r_3}\]

\noindent
\begin{itemize}
    \item Phase flip (analgously) \\
            \[E_0 = \sqrt{p} I, E_1 = \sqrt{1 - p} Z\]
            \begin{flalign*}
                \mathcal{E}_{PF}(\rho) &= \sumX_{k = 0}^1 E_k \rho E_k^\dagger \\
                &= p \cdot \rho + (1 - p) Z \rho Z
            \end{flalign*}
            \[\vec{r'} = \colvec{(2p - 1) r_1 \\ (2p - 1) r_2 \\ r_3}\]
            For \(p = \frac{1}{2}\) : by unitary freedom of Kraus representation: \\
            \begin{flalign*}
                P_0 \coloneqq \textcolor{blue}{\frac{1}{\sqrt{2}}} E_0 + \textcolor{blue}{\frac{1}{\sqrt{2}}} E_1 = \frac{1}{2} I + \frac{1}{2} Z = \colvec{ 1 & 0 \\ 0 & 0} = \ketbra[0] \\
                P_1 \coloneqq \textcolor{blue}{\frac{1}{\sqrt{2}}} E_0 - \textcolor{blue}{\frac{1}{\sqrt{2}}} E_1 = \frac{1}{2} I - \frac{1}{2} Z = \colvec{ 0 & 0 \\ 0 & 1} = \ketbra[1]
            \end{flalign*}
            \[\leadsto \mathcal{E}_{PF, \frac{1}{2}}(\rho) = P_0 \rho P_0 + P_1 \rho P_1\]
            \begin{center} (equivalent to standard basis measurement) \end{center}
    
    \item Deplolarizing channel: \\
        \begin{quote}
            Replace $\rho$ by the completely mixed state $\frac{I}{2}$ with probability p:
            \[\mathcal{E}_{DP}(\rho) = p \cdot \textcolor{forest}{\frac{I}{2}} + (1 - p) \rho\]
            Block sphere representation: uniform contraction: (images in official script) \\
            Circuit representation of deplolarizing channel: \\
            (Make circuit here) \\

            \noindent
            Kraus form of deplolarizing channel: \\
            trick: for a normalized density matrix $\rho$ \(\tr[\rho] = 1\)
            \[\textcolor{forest}{\frac{I}{2}} = \frac{\rho + X \rho X + Y \rho Y + Z \rho Z}{4}\]
            \[\leadsto \mathcal{E}_{DP}(\rho) = (1 - p + p \frac{1}{4})\rho + \frac{\rho}{4} (X \rho X + Y \rho Y + Z \rho Z)\]
            \[\leadsto E_0 = \sqrt{1 - \frac{3}{4}p} I, E_1 = \frac{\sqrt{P}}{2} X, E_2 = \frac{\sqrt{P}}{2} Y, E_3 = \frac{\sqrt{P}}{2} Z\]
        \end{quote}
    
    \item Amplitude damping (see Nielsen and Chuang section 8.3.5, Exercice 13.1) \\
        e.g. photon in cavity
        \begin{quote}
            \ketz: no photon \\
            \keto: single photon
        \end{quote}

        \[E_0 = \colvec{1 & 0 \\ 0 & \sqrt{1 - \gamma}}, E_1 = \colvec{0 & \sqrt{\gamma} \\ 0 & 0}\]
        \[E_1 \keto = \sqrt{\gamma} \ketz \ \ \ \ \text{Interpretation}: \keto \rightarrow \ketz \text{ with probability } \gamma \]

        Circuit representation:\\
        \begin{figure}[H]
            \begin{subfigure}{0.6\linewidth}
                \hspace{3em}
                \begin{quantikz}
                    \lstick{$\rho$} & \ctrl{1}              &   \targ{}     &   \rstick{$\mathcal{E}_{AD}(\rho)$}\\
                    \lstick{\ketz}  & \gate{R_y(\theta)}    &   \ctrl{-1}   &
                \end{quantikz}
            \end{subfigure}
            \hfil
            \begin{subfigure}{0.35\linewidth}
                with \(\gamma = \sinhalf{\theta}^2\)
            \end{subfigure}
        \end{figure}

    \item Phase damping (Nielsen and Chuang section 8.3.6)\\
            \begin{quote}
                "phase information" is lost \\
                model: apply a "phase kick" \(R_Z(\theta) = e^{- \frac{\theta}{2} Z} = \colvec{e^{-i \frac{\theta}{2}} & 0 \\ 0 & e^{i \frac{\theta}{2}}}\)
                with random angle $\theta \sim \frac{1}{\sqrt{4 \pi \gamma}} e^{- \frac{\theta^2}{4 \gamma}}$ (Gaussian distribution) \\ \\

                \noindent
                initial state \(\ketp = a \ketz + b \keto\) \\
                density matrix after random phase kick: 
                \begin{flalign*} 
                    \rho    &= \frac{1}{\sqrt{4 \pi \gamma}} \int_{-\infty}^{+\infty} R_Z(\theta) \ketbra R_Z(\theta)^\dagger e^{\frac{\theta^2}{4\gamma}} d\theta \\
                            &= \colvec{\abs{a}^2 & ab^* \textcolor{orange}{e^{-\gamma}} \\ a^* b \ \textcolor{orange}{e^{-\gamma}}  & \abs{b}^2}
                \end{flalign*}
                off-diagonal entities are damped by $\textcolor{orange}{e^{-\gamma}}$

                \vspace{1em}\noindent
                Kraus form: \(E_0 = \colvec{1 & 0 \\ 0 & \sqrt{1 - \lambda}}, \ E_1 = \colvec{0 & 0 \\ 0 & \sqrt{\lambda}}\)

                \noindent
                unitary freedome $\leadsto$ can show that channel is equivalent to phase channel is equivalent to phase flip channel:
                \[\tilde{E}_0 = \sqrt{\alpha} \colvec{1 & 0 \\ 0 & 1}, \tilde{E}_1 \sqrt{1 - \alpha} Z \text{ with } \alpha = \frac{1 + \sqrt{1 - \lambda}}{2}\]

                \noindent
                Application to Schrödinger's cat:
                \[\ketp = \frac{1}{\sqrt{2}} (\ket{dead} \ket{0} + \ket{alive} \ket{1})\]
                \[\rho = \ketbra = \frac{1}{2} \colvec{\ket{dead} \ket{0} \bra{dead} \bra{0} + \textcolor{orange}{\ket{dead} \ketz \bra{alive} \brao} \\ \textcolor{orange}{\ket{alive} \keto \bra{dead} \braz} + \ket{alive} \keto \bra{alive} \brao}\]
                \begin{center} \textcolor{orange}{off-diagonal entries $\rightarrow$ 0 due to decoherence} \end{center}

                \noindent
                $\rightarrow$ classical ensemble \[\frac{1}{2} (\ket{dead} \ketz \bra{dead} \braz + \ket{alive} \keto \bra{alive} \brao)\]
            \end{quote}
\end{itemize}