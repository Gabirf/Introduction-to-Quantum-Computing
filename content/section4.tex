\section{Quantum search algorithms}
(Nielsen and Chuang: section 6)\\

\noindent
Classical search through N unordered elements: O(N) \\
Quantum Groover's algorithm: O($\sqrt{N}$) (under certain preconditions)

\subsection{Quantum oracles}
search space of $N = 2^n$ elements, labelled 0, 1, $\hdots$, N-1. \\
Assume there are M solutions (with $1 \le M \le N$) \\
Define corresponding \underline{indicator function} \(f: {0, \hdots, N-1} \rightarrow {0,1}\)
\[
    f(x) = \begin{cases}
        0, & \text{if element x is \underline{not} a solution} \\
        1, & \text{if element x \underline{is} a solition}
    \end{cases}
\]

\noindent
Quantum version of f? \\
$\leadsto$ quantum "oracle" Uf defined for computational basic states

\begin{quantikz}
    \lstick{$\ket{x}$} & \qwbundle{n} & \gate[2]{U_f} && \rstick{$\ket{x}$( \textcolor{gray}{register of n quibits})}\\
    \lstick{$\ket{q}$} &              & \qw           && \rstick{$\ket{q \oplus f(x)}$}
\end{quantikz}

\noindent
Notes: \\
With \(x = x_{n-1} \hdots x_1 x_0 \Rightarrow \ket{x} = \ket{x_{n-1}} \hdots \ket{x_1} \ket{x}\).\\
$U_f$ maps basis states to basis states and satisfies $U_f^2 = I ( q \oplus f(x) \oplus f(x) = q)$ \\
Thus $U_f$ permites basis states and is in particular unitary.\\

\noindent
Initialize the oracle in superpositions $\ket{-} = \ketX[1][1]{-}$, then
\[\ket{x} \otimes \frac{\ketX{-}}{\sqrt{2}} \xmapsto{U_f} 
            \begin{cases}
                \ket{x} \otimes \frac{\ketX{-}}{\sqrt{2}} \\
                \ket{x} \otimes \frac{\keto - \ketz}{\sqrt{2}} = \textcolor{blue}{-} \ket{x} \otimes \frac{\super[-][][]}{\sqrt{2}}
            \end{cases}
\]
In summary: \(\ket{x} \otimes \frac{\ketX{-}}{\sqrt{2}} \xmapsto{U_f} \textcolor{blue}{(-1)^{f(x)}} \ket{x} \otimes \frac{\ketX{-}}{\sqrt{2}} \mid \text{(oracle qubit unchanged)}\)
Effective action of oracle:
\begin{quantikz}
    \lstick{$\ket{x}$} & \gate{U_f} & \rstick{$(-1)^{f(x)}\ket{x}$}
\end{quantikz}

$\leadsto$ oracle "marks" solutions by a phase flip. \\

\noindent
How could one construct such and oracle without knowing the solution already?\\
Example: factorisation of a large integer \(m \in N\): \\
\indent Finding prime factors of m is "difficult" on a classical computer: \\
\indent (no known algorithm with polynomial runtime in the bitlength of m) \\
But testing whether a given $x \in \mathbb{N}$ divides m is simple. \\
Can perform arithmetic operations for trial division on a digital quantum computer as well $\leadsto$ oracle which recognizes a solution x.
\textcolor{gray}{(Remark: "better" quantum algorithm for integer factorisation: Shor's algorithm.)}


\subsection{Grover's Algorithm}

Search space with $N = 2^n$ element, M solutions Overall circuit diagramm for Grover's algorithm: \\
Apply G $O(\sqrt{\frac{N}{M}})$ times.

\vspace*{10pt}
\begin{figure}[H]
    \begin{center}
        \begin{quantikz}
            \lstick{$\ket{0, \hdots, 0}$} & \qwbundle{n} & \gate{H^{\otimes n}} & \gate[2]{G} & \gate[2]{G} & \ldots & \gate[2]{G} & \meter{} & \setwiretype{c} \\
            \lstick{ow}                   &              &                      &             &             & \ldots &             &        
        \end{quantikz}
    \end{center}

    \captionsetup{
        justification=raggedright, % Aligns text to the left
        singlelinecheck=false      % REQUIRED: prevents centering of short lines
    }

    \caption*{ow: oracle workspace (auxiliary qubit)\\
    G: Groover operator\\}
\end{figure}

\begin{figure}[H]
    \begin{subfigure}[t]{0.4\textwidth}
        \vspace*{0pt}
        \noindent
        \begin{quantikz}[baseline=(current bounding box.center)]
            & \qwbundle{n} & \gate{H^{\otimes n}}
        \end{quantikz} = \begin{quantikz}[baseline=(current bounding box.center)]
            & \gate{H} & \\
            & \gate{H} & \\
            \setwiretype{n} & \vdots & \\
            & \gate{H} & 
        \end{quantikz}
    \end{subfigure}
    \hfill
    \begin{subfigure}[t]{0.5\textwidth}
        \vspace*{0pt}
        \(H = \frac{1}{\sqrt{2}} \colvec{1 & 1 \\ 1 & -1}\)
        \begin{flalign*}
            \text{Note: } & H \ketz = \frac{1}{\sqrt{2}} (\ketX{+}) \\
                          & H \keto = \frac{1}{\sqrt{2}} (\ketX{-})
        \end{flalign*}
        for \(x \in \{0, 1\} \) \\
        \(H \ket{x} = \frac{1}{\sqrt{2}} \sumX_{z = 0}^1 (-1)^{z \cdot x} \ket{z}\)
    \end{subfigure}
\end{figure}

\noindent
Applied to several qubits:
\begin{align*}
    \underbrace{H^{\otimes n}}_{H \otimes H \otimes \hdots H} \ket{x_1, \hdots, x_n} = (H \ket{x_1}) \otimes \hdots \otimes (H \ket{x_n}) 
    &\overset{(1)}{=} \frac{1}{\sqrt{2^n}} \sumX_{z_{1i} - z_n}^1 (-1)^{z_1 x_1 + \hdots + z_n x_n} \ket{z_1 \hdots z_n} \\
    &= \frac{1}{\sqrt{2}} \sumX_{z=0}^{2^n - 1} (-1)^{x \cdot z} \ket{z} \; \; \; \mid \text{z is a bitstring}
\end{align*}

\noindent
With: \\
(1): \(H \ket{x_1} = {\frac{1}{\sqrt{2}} \sumX_{z_1 = 0}^1 (-1)^{z_1 x_1} \ket{z_1}}\) \\

\noindent
In particular: \(H^{\otimes n} \ket{0,\hdots, 0} = \frac{1}{\sqrt{N}} \sumX_{z = 0}^{N - 1} \ket{z} \eqqcolon \textcolor{forest}{\text{\ketp equal superposition state}}\) \\

\newpage\noindent
Definition of Grover operator G: \\

\begin{quantikz}
    \lstick{$\ket{x}$} & \qwbundle{n} & \gate[2]{G} & \\
                       &              &             & 
\end{quantikz} = \begin{quantikz}
    & \gate[2]{U_f} & \gate{H^{\otimes n}} & \gate{ph. g.} & \gate{H^{\otimes n}} & \\
    &               & \qw                    & \qw           & \qw                    &
\end{quantikz}

\noindent
Captions: \\
$U_f$ (oracle): \(\ket{x} \mapsto (-1)^{f(x)} \ket{x}\) \\
ph. g. (phase gate): \(2 \ketz \braz - I = \begin{cases} \ketz \mapsto \ketz \\ \ket{x} \mapsto - \ket{x} for x \ne 0\end{cases}\) \\
All single qubit gates together:

\noindent
\begin{flalign*}
    H^{\otimes n} \cdot ( 2 \keto \brao - I ) \cdot H^{\otimes n} 
    &= 2 \underbrace{(H^{\otimes n} \ketz)}_{\textcolor{forest}{\ketp}} \underbrace{(H^{\otimes n} \braz)}_{\textcolor{forest}{\brap}} - I &\\
    &= 2 \textcolor{forest}{\ketp \brap} - I &
\end{flalign*}

\noindent
In summary: \(G = (2\ketp \brap - I) \cdot U_f\) \\

\vspace*{10pt} \noindent
\underline{Geometric interpretation}\\
Define:
\begin{figure}
    \begin{subfigure}{0.4\textwidth}
        \begin{flalign*}
            \ket{\alpha} &\coloneqq \frac{1}{\sqrt{N - M}} \sumX_{\substack{x = 0 \\ f(x) = 0}}^{N - 1} \ket{x} & \\
            \ket{\beta} &\coloneqq \frac{1}{\sqrt{M}} \sumX_{\substack{x = 0 \\ f(x) = 1}}^{N - 1} \ket{x} &
        \end{flalign*}
    \end{subfigure}
    \hfill
    \begin{subfigure}{0.5\textwidth}
        \includegraphics*[width=0.5\textwidth]{assets/GeoInterGrover.png}
    \end{subfigure}
\end{figure}

\noindent
Angle \textcolor{forest}{$\theta$} is defined by \(sin(\frac{\theta}{2}) = \sqrt{\frac{M}{N}}\) such that:
\[\textcolor{forest}{\ketp} = cos(\frac{\theta}{2}) \ket{\alpha} + sin(\frac{\theta}{2} \ket{\beta})\]
Note: by definition \(U_f \ket{\alpha} = \ket{\alpha}, U_f \ket{\beta} = - \ket{\beta}\) \\
$\leadsto \textcolor{purple}{U_f}$ is a \textcolor{purple}{reflection about $\ket{\alpha}$} within the subspace spanned by $\ket{\alpha}$ and $\ket{\beta}$. \\

\noindent
Likewise: \(\textcolor{forest}{2 \ketp \brap - I}\) is a \textcolor{forest}{reflection about \ketp} since \ketp is part of the subspace spanned by $\ket{\alpha}$ and $\ket{\beta}$ G leaves this subspace invariant! \\

\noindent
Thus G is a \textcolor{red}{rotation} by angle \textcolor{forest}{$\theta$}: \\
\(\ket{\phi} cos(\varphi) \ket{\alpha} + sin(\varphi) \ket{\beta}\) \\
\(\leadsto G \ket{\phi} = cos(\varphi + \textcolor{forest}{\theta}) \ket{\alpha} + sin(\varphi + \textcolor{forest}{\theta}) \ket{\beta}\) (\textcolor{gray}{see handout for algebraic derivation}) \\

\begin{itemize}[label=]
    \item For k applications of G: \\
        \hspace*{5pt} \(G^k \ket{\phi} = cos(\varphi + k \cdot \theta) \ket{\alpha} + sin(\varphi + k \cdot \theta) \ket{\beta}\)
    \item For initial state \(\ket{\phi} = \textcolor{forest}{\ketp}: \varphi = \frac{\theta}{2}\) \\
        \hspace*{5pt} \(G^k \ket{\varphi} = cos((k + \frac{1}{2}) \theta) \ket{\alpha} + sin(k + \frac{1}{2}) \theta) \ket{\beta}\)
    \item Goal: rotate to $\ket{\beta}$, i.e. \((k + \frac{1}{2} \theta) \overset{!}{=} \frac{\pi}{2}\)\\
        \hspace*{5pt} since \(sin(\frac{\theta}{2}) = \sqrt{\frac{M}{N}} \overset{\text{for M $\ll$ N}}{\rightsquigarrow} \theta \approx 2 \sqrt{\frac{M}{N}}\)
\end{itemize}

\noindent
Thus $O(\sqrt{\frac{N}{M}})$ rotations are needed $\begin{aligned}[t]
    &: k + \frac{1}{2} \overset{!}{=} \frac{\pi}{2\theta}, k \overset{!}{=} \frac{\pi}{2\theta} - \frac{1}{2} \\
    &\leadsto k = O(\sqrt{\frac{N}{M}}) \text{for N} \mapsto \infty
\end{aligned}$

\noindent
Final step: standard measurement, will collapse quantum state (with high probabilitiy) to a basis state forming $\ket{\beta}$, i.e. a solution!

\subsection{Optimality of the search algorithm}
(Nielsen and Chuang section 6.6) \\

\noindent
Goal: show that any quantum search algorithm needs $\Omega(\sqrt{N})$ oracle calls $\leadsto O(\sqrt{N})$ is already optimal. \\

\noindent
For simplicity we assume a single solution x: \\
Recall that oracle flips sign of solutions: \\
\indent \(O_x = I - 2 \ket{x} \bra{x}\) (denoted $U_f$ in the previous section)