\section{Entanglementand and its applications}
(Albert Einstein (1947): "spooky action at a distance") \\
A n-qubit state \ketp $(n \ge 2)$ is called entangled if it cannot be written as tensor product of single qubit states, 
i.e. \(\ketp = \ket{\varphi_{n-1}} \otimes \hdots \otimes \ket{\varphi_0}\) for any \(\ket{\varphi_0}, \hdots, \ket{\varphi_{n-1}} \in \mC{2}\) 

\vspace{2pt}
{\color{gray}
    \begin{quantikz}
        \lstick{$\ket{\varphi_0}$}      &&&  \\
        \lstick{$\ket{\varphi_1}$}      &&&  \\
        \lstick{\scalebox{0.75}{\vdots}}                 &&&  \\
        \lstick{$\ket{\varphi_{n-1}}$}  &&&
    \end{quantikz}
}

\noindent
Example: Bell states, also denoted EPR states (Einstein-Podolsky-Rosen)
\begin{align*}
    \ket{\beta_{00}} &= \frac{1}{\sqrt{2}}(\ketzz + \ketoo) \textcolor{gray}{ \ne \ket{a} \ket{b}} \\
    \ket{\beta_{01}} &= \frac{1}{\sqrt{2}}(\ketzo + \ketoz) \\
    \ket{\beta_{10}} &= \frac{1}{\sqrt{2}}(\ketzz + \ketoo) \\
    \ket{\beta_{11}} &= \frac{1}{\sqrt{2}}(\ketzo + \ketoz)
\end{align*}

Quantum circuit to create Bell states:

\vspace{1pt} 
\begin{quantikz}
    \lstick{$\ket{x}$} & \gate{H} & \ctrl{1} & \rstick[wires=2]{$\ket{\beta_{xy}}$ for $x, y \in {0,1}$}\\
    \lstick{$\ket{y}$} & \qw      & \targ{}  & 
\end{quantikz}


\subsection{Quantum teleportation}
(Nielsen and Chuang section 1.3.7)

\noindent
Scenario: two (experimental physicists) Alice and Bob are far away from each other.
%\includegraphics*{imagefile}
When visiting eachoter in the past, they generated the EPR pair $\ket{\beta_{00}}$, each keeping one qubit of the pair. \\
Alice's task is to send another (unknown) qubit state \ketp \,to Bob. \\
Note: measurement is not an option. \\
Quantum circuit for teleporting \ketp:

\vspace{5pt}
\begin{quantikz}
    \lstick{\textcolor{red}{\ketp}}       && \ctrl{1}  \slice{$\ket{\psi_1}$} & \gate{H} \slice{$\ket{\psi_2}$} & \meter{} & \setwiretype{c} & \wire[d][2]{c} \\
    \lstick[wires=2]{$\ket{\beta_{00}}$}  && \targ{}   \qw                    & \qw                             & \meter{} & \setwiretype{c} \wire[d][1]{c} \\
                                          && \qw       \qw                    & \qw                             & \qw      & \gate{x}        & \gate{z}       & \rstick{\textcolor{red}{\ketp}}
\end{quantikz} \\
\textcolor{gray}{(wire 1 \& 2 represent Alice's 2 qubits, wire 3 represents Bob's Qubit.)} \\

\noindent
Input: \begin{align*} \ketp \ket{\beta_{00}} = \ketp \otimes \ket{\beta_{00}} &= (\super) \otimes (\ketzz + \ketoo) \\ &= \frac{1}{\sqrt{2}} (\alpha \keto(\ketzz + \ketoo) + \beta\keto (\ketzz + \ketoo))\end{align*}

\noindent
after CNOT: \\
\(\frac{1}{\sqrt{2}} (\alpha \keto(\ketzz + \ketoo) + \beta\keto (\ket{\textcolor{red}{1}0} + \ket{\textcolor{red}{0}1}))\) \\

\noindent
after Hadamard:
\begin{align*}
    \ket{\psi_2} 
    &= \frac{1}{\sqrt{2}} (\alpha (\ketX{+})(\ketzz + \ketoo) + \beta(\ketX{-}) (\ketoz + \ketzo)) \\
    &= \frac{1}{2} (\alpha \ket{000} + \alpha\ket{011} + \alpha\ket{101} + \alpha\ket{111} + \beta\ket{010} + \beta\ket{001} - \beta\ket{110} - \beta\ket{101}) \\
    &= \frac{1}{2} \ketzz (\super) + \ketzo (\alpha\keto + \beta\ketz) + \ketoz (\super[-]) + \ketoo(\alpha\keto - \beta\ketz)
\end{align*}

\noindent
Now Alice measures her qubits w.r.t computational basis.\\
Projective measurement with 
\begin{align*}
    P_1 &= \ketzz \brazz \otimes I, \; P_2 = \ketzo \brazo \otimes I,\\ P_3 &= \ketoz \braoz \otimes I,\; P_4 = \ketoo \braoo \otimes I
\end{align*}
If Alice measures 00, then $\ket{\psi_2}$ will collapse to:

\[
    \ketzz (\super) = \ketzz \textcolor{red}{\ketp \text{(qubit at bobs place)}}
\]

similary:
\begin{align*}
    00 &\longrightarrow \super \\
    01 &\longrightarrow \alpha\keto + \beta\ketz \\
    10 &\longrightarrow \super[-] \\
    11 &\longrightarrow \alpha\keto - \beta\ketz \\
\end{align*}

\noindent
Alice transmits her measurement results to Bob (classical information). \\
Bob then applies Pauli-X and/or Pauli-Z if necessary to recover the original state \ketp.

\noindent
Even though the wavefunction collapse is instantanious, it does not allow for "faster-than-light" information transfer due to the required classical communication and the randomness of the measurement outcome.

\newpage
\subsection{EPR and the Bell inequality}
(Nielsen and Chuang section 2.6) \\
(Albert Einstein (1926): "god does not play dice")\\

\noindent
EPR (Einstein-Podolsky-Rosen) paper: \\ "Can quantum mechanical description of physical reality be considered complete?" \\

\noindent
The authors argue that quantum mechanics is incomplete, since it lacks certain "elements of reality" (properties that can be predicted with certainty).\\

\noindent
Scenario: Alice and Bob are far away from eachother but share the entangled two-qubit "spin-singlet" state 

\(\ket{\beta_{11}} = \frac{1}{\sqrt{2}} (\ketzo - \ketoz)\) \\
Alice and Bob measure the observable $\vec{v} \circ \vec{\sigma} = v_1X + v_2Y + v_3Z$ \\
(with $\vec{v} \in \mathbb{R}^3, \norm{\vec{v}} = 1$) on their respective qubit. \\
Recall that $\vec{v} \circ \vec{\sigma}$ is Hermitian and unitary and has eigenvalues $\pm 1$. \\
Alice performs her measurement immediatly before Bob.\\

\noindent
Example:

\begin{itemize}
    \item $\vec{v} = (0, 0, 1),$ observable Z = $\textcolor{blue}{1} \cdot \ketz\braz + \textcolor{green}{(-1)} \cdot \keto\brao$  $\mid$ (standard measurement) \\
        if Alice measures the eigenvalue:
        \begin{enumerate}[label=]
            \item  \ \textcolor{blue}{1}: wavefunction collapses to \ketzo
            \item \textcolor{green}{-1}: wavefunction collapsed to \ketoz
        \end{enumerate}
        $\leadsto$ Bob will always obtain the opposite measurement result.
    
    \item $\vec{v} = (1, 0, 0)$: observable X, eigenstates $\ket{\pm} = \frac{1}{\sqrt{2}}(\super[\pm])$\\
            (measurement w.r.t. ${\ket{+}, \ket{-}}$ basis) \\
            We can represent the wavefunction as:
            \[\ket{\beta_{11}} = \frac{1}{\sqrt{2}} (\ket{+-} - \ket{-+})\]
            namely: 
            \begin{align*}
                -\frac{1}{\sqrt{2}} (\ket{+-}\; - \;\ket{-+}) 
                &= - \frac{1}{\sqrt{2}} (\;\frac{1}{2}(\super) (\super[-]) - \frac{1}{2}(\super[-]) (\super)) \\
                &= \hdots \\
                &= \frac{1}{\sqrt{2}}(\ketzo - \ketoz) \\
                &= \ket{\beta_{11}}
            \end{align*}

            \noindent
            If Alice measures eigenvalue 1, the wavefunction will collapse t $\ket{+-} \leadsto$ Bob'squbit is in state $\ket{-}$.\\
            So he will certainly measure eigenvalue (-1). (conversly if Alice measures -1)
    
    \item general observable $\vec{v} \circ \vec{\sigma}$, general unit vector $\vec{v} \in \mathbb{R}^3:$ denote the orthogonal eigenstates of $\vec{v} \circ \vec{\sigma}$ by $\ket{a}, \ket{b},$ then there exist complex numbers $\alpha, \beta, \gamma$ and $\delta$ such that:
            \begin{align*}
                \ketz &= \alpha\ket{\alpha} + \beta\ket{\beta} \\
                \keto &= \gamma\ket{\alpha} + \delta\ket{\beta}
            \end{align*}
            Inserted into $\ket{\beta_{11}}$ (see also Exercice 8.1(a)):
            \[\frac{1}{\sqrt2} (\ketzo - \ketoz) = \underbrace{(\alpha\delta - \beta\gamma)}_{det(U)\text{ with U }= \colvec{\alpha & \beta \\ \gamma & \delta}} \frac{1}{\sqrt{2}}(\ket{ab} - \ket{ba})\]
            U is a base change matrix between the orthonormal \compbas and $\{\ket{a}, \ket{b}\}$ basis $\leadsto$ U unitary $\rightarrow |det(U)| = 1$ (Ex. 1.2(e))
            Can represent det(U) = \(e^{i\theta}, \theta \in \mathbb{R}\) \\
            In summary: \(\frac{1}{\sqrt{2}} (\ketzo - \ketoz) = e^{i\theta}\frac{1}{\sqrt{2}} (\ket{ab} - \ket{ba})\)\\
            $\leadsto$ as before: Bob will obtain opposite measurement result as Alice.\\
            Therefore Alice can predict Bob's measurement result.
\end{itemize}

\noindent
However, there is no possibility that alice could influence Bob's measurement (after performing her measurement) since they are far apart (speed of light is too slow) \\
EPR argument: "property" $\vec{v} \circ \vec{\sigma}$ of a qubit is an "element of reality". \\
Quantum mechanics does not a priori specify this property for all possible $\vec{v}$ (but only probabilities), and is thus an incomplete description of reality. \\

\noindent
Instead: "hidden variable theory": there must be additional variables "hidden" in a qubit which determines Bob's measurement of $\vec{v} \circ \vec{\sigma}$ for all possible $\vec{v} \in \mathbb{R}^3$. \\

\noindent
(Niels Bohr (October 1927): "Stop telling God what to do.")\\
Bell's inequality: experimental test which can \underline{invalidate} local hidden variable theories. (Bell 1964)\\

\noindent
"\underline{local}": no faster-than-light communication possible (otherwise one could send information backwards in time according to special relativity)

\noindent
Experimental schematic: many repetitions (to collecect satistics of the following setup): \\
"TODO import image"

\begin{itemize}
    \item Charlie: prepares two particles one for Alice and one for Bob.
    \item Alice: decides randomly whether to measure Q or B.
    \item Bob: decides randomly whether to measure S or T.
\end{itemize}

\noindent
Binary property values: \(Q \in {\pm 1}, R \in {\pm 1}, S \in {\pm 1}, T \in {\pm 1}\).\\
Alice and Bob perform their measurement (almost) simultaniously, such that no information about the result can be transmitted in between them.\\
After this protocol, Alice and Bob meet to analyze their measurement data. \\
Consider the quantity:
\[Q \cdot S + R \cdot S + R \cdot T - Q \cdot T = \underbrace{(Q + R)}_{\pm 2 \; \lor \; 0} \cdot S + \underbrace{(R - Q)}_{0 \; \lor \; \pm 2}\cdot T = \pm 2\]
Denote by p(q,r,s,t) the proability that the syste before this measurements is in the state \{Q = q, R = r, S = s, T = t\}, then we compute the average value:
\begin{align*}
    \expec[Q \cdot S + R \cdot S + R \cdot T - Q \cdot T] &= \sumX_{p, r, s, t \in {\pm 1}} p(q, r, s, t) \cdot \underbrace{(1 \cdot s + r \cdot s + r \cdot t - q \cdot t)}_{\pm 2} \\ &\le \sumX_{p, r, s, t \in {\pm 1}} p(q, r, s, t) \cdot 2 = 2
\end{align*}
By linearity of \expec[], we arrive at the following: \\

\underline{Bell inequality:} \\
\[\expec[Q \cdot S] + \expec[R \cdot S] + \expec[R \cdot T] + \expec[Q \cdot T] \le 2\]
Each term can be experimentally evaluated, e.g. \expec[Q \cdot S]: Alice and Bob average over cases where Alice measured Q and Bob measured S. \\

\noindent
Compare with a "quantum" realization of the experiment: \\
\indent Charlie prepares the two-qubit singlet state: \\
\[\ketp = \frac{1}{\sqrt{2}} (\ket{\textcolor{green}{0} \textcolor{orange}{1}} - \ket{\textcolor{green}{1} \textcolor{orange}{0}})\]
\indent and sends the first qubit to \textcolor{green}{Alice} and the 2nd to \textcolor{orange}{Bob}. \\

Observables:
\begin{align*}
    Q &= Z_1 \; ( _1 \rightarrow\; first\; qubit\; (Alice)) \hspace{10pt} 
      & S &= \frac{-Z_2 - X_2}{\sqrt{2}} \;( _2 \rightarrow\; second\; qubit\; (Bob))\\
    R &= X_1 
      & T &= \frac{Z_2 - X_2}{\sqrt{2}} 
\end{align*}

\noindent
Measurement averages (c.f. Exercise 8.1)\\
\(\Braket{Q \cdot S} = \BraKet{Q \otimes S} = \frac{1}{\sqrt{2}},\hspace{20pt} \Braket{R \cdot S} = \frac{1}{\sqrt{2}}\) \\
\(\Braket{R \cdot T} = \frac{1}{\sqrt{2}},\hspace{90pt} \Braket{Q \cdot T} = - \frac{1}{\sqrt{2}}\) \\ \\
\(\Braket{Q \cdot S } + \Braket{R \cdot S} + \Braket{R \cdot T} - \Braket{Q \cdot T} = 2 \sqrt{2} \; \textcolor{red}{\nleq 2}\) (violates Bell's inequality!)\\

\noindent
Actual laboratory experiments using photons agree with predictions by quantum mechanics thus not all (implicit) assumptions leading to Bell's inequality can be satisfied.
\begin{itemize}
    \item "realism": physical properties Q, S, R, T have definit values independent of observation (measurement)
    \item "locality": Alice performing her measurement cannot influence Bob's measurementand vice versa.
\end{itemize}
$\leadsto$ Nature is not "locally realistic" (most common viewpoint: realism does not hold) \\
Practical lesson: we can use entanglement as a ressource.\\

\noindent
(John Wheeler: If you are not completely confused by quantum mechanics, you do not understand it.)
