\section{Section 2}
\subsection{Section 2.1}
(Nielsen and Chuang section 1.2)\\

\noindent
Classical bits: 0, 1 \\
Quantum bits "qubits": superposition of 0 and 1 \\

\noindent
A quantum state \ketp is described as
\[ \ketp = \super, \alpha, \beta \in \mathbb{C}\]
with \(|\alpha|^2 + |\beta|^2 = 1\) (normalisation) \\

\noindent
ket-notation: \ketp \; (motivation from inner product) \\
Mathematical description: \(\ketp \in \mathbb{C}^2\), with
\[\ketz = \colvec{1 \\ 0}, \; \keto = \colvec{0 \\ 1}, \; \ketp = \colvec{\alpha \\ \beta}\]

\noindent
Different from a classical bit, one cannot (in general) directly ovserve/measure a qubit (the amplitudes $\alpha$ and $\beta$).\\\\
Instead "standard measurement" will result in:
\begin{flalign*}
    \hspace*{25pt}  & 0 \text{ with probability } |\alpha|^2. &\\
                    & 1 \text{ with probability } |\beta|^2.  &
\end{flalign*}

\noindent
The measurement also \underline{changes} the qubit ("wavefunction collapse"):\\
If meausuring 0, the qubit  will be \(\ketp = \ketz\) directly after the measurement, and likewise if measuring 1, the
qubit will be \(\ketp = \keto\). \\

\noindent
In practice: One can estimate the probabilities $|\alpha|^2$ and $|\beta|^2$ in experiments by replicating the same experiment many times (i.e. via outcomes satistics). \\
The repetitions are called "trials" or "shots". \\
Circuit notation:
\begin{quantikz}
    \lstick{\textcolor{forest}{\ketp}} & \meter{} & \setwiretype{c} && \wire[l][1]["\text{classical info}"{above}]{c} &
\end{quantikz} \begin{quantikz}
    \lstick{\textcolor{forest}{\ketp}} & \meter{} & \rstick{\textcolor{forest}{collapsed}}
\end{quantikz}

\subsection{Single qubit gates}
(Nielsen and Chang sections 1.3.1, 2.1.8, 4.2) \\
Principle of \underline{time evolution}: the quantum state \tket{\psi} at current time point t transitions to a new quantum state \tket{\psi'} at a later time t' $\ge$ t. \\
The transition will always be described by a complex unitary matrix U.
\[
    \ket{\psi'} = U \cdot \ket{\psi}
\]
Circuit notation: \\
\begin{center}
    \begin{quantikz}
        \lstick{\ketp} && \gate{U} && \rstick{\ketpp}
    \end{quantikz} 
\end{center}

\noindent Notes: 
\begin{itemize}
    \item The circuit is read from left to right, but the matrix time vector (U \ketp ) from right to left.
    \item U preserves normalisation.
\end{itemize} 
\newpage
\noindent Examples:
\begin{itemize}
    \item quantum analogue of classical NOT-gate (0 $ \leftrightarrow $ 1) flip $ \ket{0} \leftrightarrow \ket{1} \\ \leadsto $ Pauli-X gate:

    \[
        X \equiv \sigma_1 = \begin{pmatrix} 0 && 1 \\ 1 && 0 \end{pmatrix}
    \]
    
     (check: $ x \; \ketz = \unimattwo \colvec{1 \\ 0} = \colvec{0 \\ 1} = \keto$ ) \\ analogue for 1 to 0.

    \item Pauli-Y gate:\[Y \equiv \sigma_2 = \colvec{0 & -i \\ i & 0}\]
    \item Pauli-Z gate: \[Z \equiv \sigma_3 = \colvec{1 & 0 \\ 0 & -1}\]
    Z leaves \ketz \ unchanged, but flips the sign of the coefficient of \keto.

    Recall the block sphere representation of a general quantum state:
    \[
        \ketp = cos(\frac{\varphi}{2}) \ketz + e^{i\varphi} sin(\frac{\varphi}{2}) \keto
    \]
    Then: 
    \begin{align}
        Z\ketp &= cos(\frac{\varphi}{2}) \ketz \textcolor{red}{-} \ e^{i\varphi} sin(\frac{\varphi}{2}) \keto \\ 
        &= cos(\frac{\varphi}{2})\ketz + e^{i(\varphi + \pi)} sin(\frac{\varphi}{2}) \keto  && \text{(with $e^{i\pi} = -1$)}
    \end{align}
    $\leadsto$ new Block sphere angles: $\theta' = \theta, \ \varphi' = \varphi + \pi$ \\
    (rotation by $\pi$ = $180\deg$ around z-axis) \\ 
    x, y and z  gates are called Pauli matrices. \\
    The \underline{Pauli vector} $\Vec{\sigma} = (\sigma_1, \sigma_2, \sigma_3) = (x,y,z)$ is a vector of 2x2 matrices.

    \item Hadamard gate:
    \[ H = \frac{1}{\sqrt{2}} \colvec{1 & 1 \\ 1 & -1}\] \\
    \[\alpha \ketz + \beta \keto \qc{H} \alpha \frac{\ketz + \keto}{\sqrt{2}} + \beta \frac{\ketz - \keto}{\sqrt{2}}\]

    \item Phase gate:
    \[ S = \colvec{1 & 0 \\ 0 & i}\]
    \item T gate:
    \[ T = \colvec{1 & 0 \\ 0 & e^{i\frac{\pi}{4}}}\]
    Note: $T^2 = S (\text{since }(e^{i\frac{\pi}{4}})^2 = e^{i\frac{\pi}{2}} = i) ; T^4 = S^2 = Z$
\end{itemize}

\vspace{10pt}

\noindent Pauli matrices satisfy: 
\begin{itemize}[label={}]
    \item $\sigma_j^2 = identity$.
    \item $\sigma_j \cdot \sigma_k = - \sigma_k \cdot \sigma_j$ for all $j \ne k$
    \item Commutator: $[ \sigma_j, \sigma_k] := \sigma_j \sigma_k - \sigma_k \sigma_j = 2i \ \sigma_l \hspace{15pt} \text{for (j, k, l) --- which is a cyclic permutation of (1,2,3)}$
\end{itemize}
Matrix exponential and rotation gates $ R_x(\theta) \; R_y(\theta), R_z(\theta) $ (pdf in Moodle)
Z-Y decomposition of an arbitrary 2x2 matrix: \\
For any unitary matrix $U \in \mathbb{C}^{2x2}$ there exists real numbers $\alpha, \beta, \gamma, \delta \in \mathbb{R}$ such that:
\[ U = e^{i\alpha} \; \cdot \; R_z(\beta) \; \cdot \; R_y(\gamma) \; \cdot \; R_z(\delta) = e^{i\alpha} \Rz{\beta} \cdot \Ry{\gamma} \cdot \Rz{\delta}\]

\subsection{Multiple qubit}
(Nielsen and Chang sections 1.2.1, 2.1.7)\\

\noindent
So far: single qubits, superposition of basis states \ketz and \keto. \\
For two qubits, this generalises to: \\
\indent ${\ket{00}, \ket{01}, \ket{10}, \ket{11}}$ \\
as computational basis states: all combinations (bitstrings) of 0s and 1s.

\noindent
General two-qubit state: 
\[ \ketp = \alpha_{00} \ket{00} + \alpha_{01}\ket{01} + \alpha_{10}\ket{10} + \alpha_{11}\ket{11}\]
with amplitudes $\alpha_{ij} \in \mathbb{C}$ such that:
\[|\alpha_{00}|^2 + |\alpha_{01}|^2 + |\alpha_{10}|^2 + |\alpha_{11}|^2 = 1 \hspace{10pt} \text{(normalization)}\]

\noindent
Can identify the basis states with unit vectors:
\[\ket{00} = \colvec{ 1 \\ 0 \\ 0 \\ 0}, \ket{01} = \colvec{ 0 \\ 1 \\ 0 \\ 0}, \ket{10} = \colvec{ 0 \\ 0 \\ 1 \\ 0}, \ket{11} = \colvec{ 0 \\ 0 \\ 0 \\ 1}\]

\noindent
Thus:
\[\ketp = \colvec{\alpha_{00} \\ \alpha_{01} \\ \alpha_{10} \\ \alpha_{11}} \in \mathbb{C}^4\]

\vspace{5pt}\noindent
What happens if we measure only one qubit of a two-qubit state? \\
Say we measure the first qubit: obtain the result.
\begin{itemize}[label= {}]
    \item 0 with probability $|\alpha_{00}|^2 + |\alpha_{01}|^2$
    \item 1 with probability $|\alpha_{10}|^2 + |\alpha_{11}|^2$
\end{itemize}

\noindent
Wavefunction directly after measurement:
\begin{itemize}[label={}]
    \item if we measured 0: $\ketpp = \frac{\alpha_{00} \ket{00} + \alpha_{01}\ket{01}}{\sqrt{|\alpha_{00}|^2 + \; |\alpha_{01}|^2}}$
    \item if we measured 1: $\ketpp = \frac{\alpha_{10} \ket{10} + \alpha_{11}\ket{11}}{\sqrt{|\alpha_{10}|^2 + \; |\alpha_{11}|^2}}$
\end{itemize}

\noindent
Mathematical formalism for constructing two-qubit states: \\
Tensor product of vector spaces can combine two (arbitrary) vector spaces V and W to form the \underline{tensor product} $V \otimes W$. \\(Look into the "Tensor products of vector spaces" --- Moodle) \\

\noindent
Generalisation to n qubits: \textcolor{blue}{$2^n$} computational basis states \\
$\{\ket{0, \hdots, 0}, \ket{0, \hdots, 0, 1}, \ket{0, \hdots, 1, 0}, \hdots, \ket{1, \hdots, 1}\}$ (all bit strings of length n) \\
Thus, a general n-qubit quantum state, also denoted as "quantum register", is given by:
\begin{align}
    \ketp &= \sum_{x_0 = 0}^1 \sum_{x_1 = 0}^1 \hdots \sum_{x_{n-1} = 0}^1 \alpha_{x_{n-1}, \hdots, x_1, x_0} \ket{x_{n-1} \hdots x_1 x_0}\\
    &= \sum_{x = 0}^{2^n-1} \alpha_x \ket{x} && \text{binary representation}
\end{align}
with $\alpha_x \in \mathbb{C}$ for all $x \in \{0, \hdots, 2^n-1\}$ such that: \[\norm{\ketp}^2 = \sumX_{x = 0}^{2^n-1} |\alpha_x|^2 \overset{!}{=} 1 \hspace{10pt}\text{(normalisation)}\]
$\leadsto$ in general "hard" to simulate on a classical computer (for large n) due to this \textcolor{orange}{"curse of dimensionality"}. \\

\noindent
Vector space as tensor products: $\underbrace{\mathbb{C}^2 \otimes \hdots \otimes \mathbb{C}^2}_{\text{n times}} = (\mathbb{C}^2)^{\otimes n} \approx \mathbb{C}^{(2^n)} $ \\


\newpage
\subsection{Multiple qubit gates}
(Nielsen and Chuang sections 1.3.2, 1.3.4, 2.17)\\

\noindent
As for single qubits, an operation on multiple qubits is described by an unitary matrix U. \\
For n qubits: $U \in \mC{2^n \times \, 2^n}$ \\

\noindent
Example:
\underline{Controlled-NOT} gate (also denoted CNOT) \\
two qubits: control and target \\ 
target qubit gets flipped if control is 1:
\begin{center}
    $\underbrace{\ket{\textcolor{orange}{0}0}}_{\textcolor{orange}{control} | target} \mapsto \hspace{15pt}\ket{00}$,\\ \vspace{5pt}
    $\ket{\textcolor{orange}{0}1} \mapsto \ket{01}$, $\ket{\textcolor{orange}{1}0} \mapsto \ket{1\textcolor{red}{1}}$, $\ket{\textcolor{orange}{1}1} \mapsto \ket{1\textcolor{red}{0}}$
\end{center}

\noindent
Can be expressed as:
\[\ket{ab} \mapsto \ket{a, a \oplus b}, \forall a, b \in \{0, 1\} \hspace{15pt} | a \oplus b \text{ defined as "addition modulo 2" }\]

\noindent
Circuit notation: \medskip

\begin{quantikz}
    \lstick{\tket{a}} && \ctrl{1} && \rstick{\tket{a}} \\
    \lstick{\tket{b}} && \targ{} && \rstick{\tket{a \oplus b}}
\end{quantikz}

\noindent
Matrix Representation: \\
\[
    U = 
    \begin{pNiceMatrix}
        1 & 0 & 0 & 0 \\
        0 & 1 & 0 & 0 \\
        0 & 0 & 0 & 1 \\
        0 & 0 & 1 & 0
        
        \CodeAfter
        \tikz \draw[red, thick,]
        (3-|3) rectangle (5-|5);
    \end{pNiceMatrix}
    \textcolor{red}{Pauli-X}
\]
This matrix is unitary.\\
Alternative notation: 
\[
    \begin{quantikz}
        & \ctrl{1} & \\
        & \targ{} &
    \end{quantikz} =
    \begin{quantikz}
        & \ctrl{1} & \\
        & \gate{X} & 
    \end{quantikz}
\]
We can generalise Pauli-X to any unitary single-qubit gate U acting on the target qubit $\leadsto$ controlled U-gate:
\[\ket{\textcolor{orange}{0}0} \mapsto \ket{00}, \ket{01} \mapsto \ket{01}, \ket{10} \mapsto \ket{1}\textcolor{blue}{\otimes}(U\ket{0}), \ket{11} \mapsto (\ket{1}\textcolor{blue}{\otimes}(U\keto)\]

\newpage
Generalisation:
\[
    \begin{quantikz}
        && \ctrl{1} && \\
        && \gate{U} &&
    \end{quantikz} \overset{\wedge}{=}
    \begin{pmatrix}
        1 & & \\
        & 1 & \\
        & & U
    \end{pmatrix}
\]
Example: controlled-z:
\[
    \begin{quantikz}
        && \ctrl{1} && \\
        && \gate{Z} &&
    \end{quantikz} \overset{\wedge}{=}
    \begin{pmatrix}
        1 & & & \\
        & 1 & & \\
        & & 1 & \\
        & & & -1
    \end{pmatrix}
\]
Exercice: show that controlled-z gate is invariant when flipping control and target qubits. \\
Controlled-U gate for multiple target qubits: \hspace{10 pt}
\begin{quantikz}
    && \ctrl{1}             && \\
    && \gate[wires=3]{U}    && \qw \\
    &&                      && \qw \\
    &&                      && \qw
\end{quantikz} \\

\noindent
Note: single qubit and CNOT gates are universal: they can be used to implement an arbitrary unitary operation on n qubits. \\
(Quantum analogue of universality of classical NAND gate) \\
proof in Nielsen and Chang section 4.5

\noindent
Example of a circuit consisting only of single qubit gates and CNOTs: \\
\begin{center}
    \begin{quantikz}
        & \gate{H} & \targ{}   & \qw                        & \qw      & \ctrl{1} & \qw \\
        & \gate{X} & \ctrl{-1} & \gate{R_y(\frac{\pi}{3})}  & \ctrl{1} & \targ{}  & \qw \\
        & \qw      & \qw       & \qw                        & \targ{}  & \qw      & \gate{Z}
    \end{quantikz}

(time left to right)
\end{center}

\noindent
Matrix Kronecker products: matrix representation of single qubit gates acting in parallel:
\begin{align*}
    \qc{A} \\
    \qc{B}
\end{align*}

\newpage \noindent
Operation on basis states: $a, b \in \{0, 1\}$
\[\underbrace{\ket{a, b}}_{\ket{a} \textcolor{blue}{\otimes} \ket{b}} \mapsto (A\ket{a}) \textcolor{blue}{\otimes} (B\ket{b}) \textcolor{gray}{ = (A\otimes B) \ket{a, b}}\] 

\noindent
Example: A = I (identity), B = Y: \\
\[\ketzz \mapsto \ketz \otimes \underbrace{(Y\ketz)}_{i\keto} = \textcolor{purple}{i \ketzo}\]\\
\[\ketzo \mapsto \ketz \otimes \underbrace{(Y\keto)}_{-i\ketz} = \textcolor{green}{-i \ketzz}\]\\
\[\ketoz \mapsto \keto \otimes (Y\ketz) = i \ketoo\]\\
\[\ketoo \mapsto \keto \otimes (Y\keto) = -i \ketoz\]

\noindent
Matrix Representation: \\
(ket on the left of mapto represents column and resulting one on the row) \\

\begin{center}
    \begin{quantikz}
        && \gate{I} && \\
        && \gate{Y} && 
    \end{quantikz} $\overset{\wedge}{=}$
    $\begin{pNiceMatrix}
        0 & \textcolor{green}{-i} & 0 & 0 \\
        \textcolor{purple}{i} & 0 & 0 & 0 \\
        0 & 0 & 0 & -i \\
        0 & 0 & i & 0

        \CodeAfter
        \tikz \draw[red, thick,]
        (3-|3) rectangle (5-|5);
        \tikz \draw[red, thick,]
        (1-|1) rectangle (3-|3);
    \end{pNiceMatrix} \textcolor{red}{Pauli-Y} = 
    \begin{pmatrix}
        Y & 0 \\
        0 & Y \\
    \end{pmatrix} = I \otimes Y$
\end{center}
General formula: \underline{Kronecker product} (matrix representation of tensor products of operators)

\noindent
\[
    A \otimes B =
    \begin{pmatrix}
        a_{11}B & a_{12}B & \hdots & a_{1n}B \\
        a_{21}B & a_{22}B & \hdots & a_{2n}B \\
        \vdots  & \vdots  & \vdots & \vdots  \\
        a_{m1}B & a_{m2}B & \hdots & a_{mn}B 
    \end{pmatrix} \in \mC{mp \times nq}
\] $\forall A \in \mC{m\times n}, B \in \mC{p \times q}$ (NumPy np.kron(A, B))

\noindent
Generalise to arbitrary number of tensor factors, e.g. \\
\[
    \begin{quantikz}
        && \gate{A} && \\
        && \gate{B} && \\
        && \gate{C} &&
    \end{quantikz} \overset{\wedge}{=} A \otimes B \otimes C \textcolor{gray}{\; = (A \otimes B) \otimes C = A \otimes (B \otimes C)}
\] 

\noindent \newpage
Basic properties:
\begin{enumerate}[label=(\alph*)]
    \item $(A \otimes B)^* = A^* \otimes B^*$ (elementwise complex conjugation)
    \item $(A \otimes B)^T = A^T \otimes B^T$ (transposition)
    \item $(A \otimes B)^\dagger = A^\dagger \otimes B^\dagger$
    \item $(A \otimes B) \otimes C = A \otimes (B \otimes C)$ (associative property)
    \item $\underbrace{(A \otimes B) \cdot (C \otimes D)}_{\text{matrix-matrix multiplication}} = (A \cdot B) \otimes (B \cdot D) $ \\
    
    for matrices of compatible dimensions.\\
    
    \vspace{5pt}
    \begin{quantikz}
        & \gate{C} && \gate{A} & \\
        & \gate{D} && \gate{B} & 
    \end{quantikz} = 
    \begin{quantikz}
        && \gate{A \cdot C} && \\
        && \gate{B \cdot D} &&
    \end{quantikz} \\ (side exchange because its read from left to right)

    \item Kronecker product of Hermitian matrices is Hermitian
    \item Kronecker product of unitary matrices is unitary (follows from (c) \& (e))
\end{enumerate}

\subsection{Quantum measurements}
(Nielsen and Chuang sections 1.3.3, 2.2.3, 2.2.5) \\
Review: measurement of a single qubit \ketp = \super with respect to computational basis{\ketz, \keto}:

\begin{quantikz}
    \lstick{\ketp} && \meter{} & \setwiretype{c} &
\end{quantikz} classical data (measurement outcome 0 or 1) \\

\noindent
alternative notation:

\begin{quantikz}
    \lstick{\ketp}              & \meter{} \vcw{1}  & \rstick{qubit after measurement (collapsed wavefunction)}\\
    \lstick{C} \setwiretype{c}  &                   & \rstick{measurement outcome}
\end{quantikz} \\

\newpage \noindent
\underline{Unitary freedom of choice of measurement basis.} \\
Given an orthonormal basis (ONB) $\{\ket{u_1}, \ket{u_2}\}$, we can measure with respect to this orthonormal basis
by performing a base change before and after the measurement: \\
\(
    U = (\ket{u_1} \ket{u_2}) \in \mC{2 \times 2} \; \text{unitary}
\) \\ 
\begin{quantikz}
    \lstick{\ketp}              & \gate{u^\dagger} & \meter{} \vcw{1} & \gate{u}  & \\
    \lstick{C} \setwiretype{c}  &            &                  &           &
\end{quantikz} \\
(measurement with respect to $\{\ket{u_1, \ket{u_2}}\}$, using a standard basis measurement) \\

\noindent
Representing the qubit \ketp = \(\alpha_1 \ket{u_1} + \alpha_2 \ket{u_2}\), $\alpha_1, \alpha_2 \in \mC{}$

\begin{quantikz}
    \lstick{\ketp} & \gate{U^{\dagger}} &
\end{quantikz} \(\leadsto U^{\dagger} \ketp = \super[\alpha_1][\alpha_2] \hspace{5pt} \textcolor{gray}{(U^{\dagger}\ket{u_1} = \ketz, U^{\dagger}\ket{u_2} = \keto)}\)

\noindent
We will obtain measurement result 0 or 1 with the respective probabilities $|\alpha_1|^2 \text{ and } |\alpha_2|^2$ \\
After measuring and applying U: \ketp  will be in the state $\ket{u_1}$ or $\ket{u_2}$. \\

\noindent
\underline{Abstract, general defenition of quantum measurements}\\
Quantum measurements are described by a collection $\{M_m\}$ of \underline{measurement operators} acting on the quantum system,
with the index m labelling possible measurement outcomes. \\
Denoting the quantum state before the measurement by \ketp, result m occurs with probability:
\[p(m) = \brap M_m^\dagger M_m \ketp = \norm{M_m \ketp}^2,\]
state after the measurement is:
\[\frac{M_m\ketp}{\norm{M_m\ketp}}\]
The measurement operators satisfy the \underline{completeness relation}:
\[\sumX_m M_m^\dagger M_m = I,\]
such that the probabilities sum to 1: \\



\begin{align*}
    \sumX_m p(m) &= \sumX_m \BraKet{\underbrace{\sumX_m M_m^\dagger M_m}_I} = \BraKet{} = 1
\end{align*}

\newpage \noindent
Example: measurement of a qubit \ketp = \super with respect to computational basis \compbas:
\begin{align*}
    M_0 &\coloneqq \underbrace{\ketz \braz}_{\text{outer product}} = \colvec{1 \\ 0} \colvec{1 & 0} = \colvec{1 & 0 \\ 0 & 0} \\
    M_1 &\coloneqq \keto \brao = \colvec{0 \\ 1} \colvec{0 & 1} = \colvec{0 & 0 \\ 0 & 1}
\end{align*}
\begin{align*}
    \leadsto & p(0) = \BraKet{M_0^\dagger M_0} = \BraKet{M_0} &&= \colvec{\alpha^* & \beta^*} \begin{pmatrix} 1 & 0 \\ 0 & 0 \end{pmatrix} \colvec{\alpha \\ \beta} \notag \\
             & &&= |\alpha|^2 \\
             & p(1) = \BraKet{M_1^\dagger M_1} = \hdots &&= |\beta|^2
\end{align*}

\noindent
\underline{Projective measurements} \\
\indent \textcolor{gray}{$\leadsto$ see projection-operators (moodle)} \\
Definition: a \underline{projective measurement} is described by an \underline{observable} M, a Hermitian operator acting on the quantum system spectral decomposition $\leadsto$
\[M = \sumX_m \lambda_m \textcolor{blue}{P_m}\]
with \textcolor{blue}{$P_m$}: projection onto the eigenspace with eigenvalue $\lambda_m$. \\
The possible outcomes of the measurement correspond to the eigenvalues $\lambda_m$. \\
Probability of obtaining measurement result $\lambda_m$ :
\[
    \textcolor{orange}{p(}\lambda_m \textcolor{orange}{)} = \BraKet{\smash[b]{
    \textcolor{blue}{P_m}}} \hspace{10pt} |\textcolor{gray}{P_m^\dagger P_m = P_m \text{ since } P_m \text{ is a projection}}
\]
State of the quantum system after the measurement:
\[
    \frac{P_m\ketp}{\norm{P_m\ketp}} = \frac{P_m\ketp}{\sqrt{\textcolor{orange}{p(}\lambda_m\textcolor{orange}{)}}}.
\]

\noindent
Remarks:
\begin{itemize}
    \item Projective measurements are special cases of a general measurement framework
    \item Projective measurement combined with transformations (and auxiliary qubit) are equivalent to the general measurement formalism/framework.
          (Proof: see pages 94, 95 in Nielsen and Chuang)
\end{itemize}

\noindent
Average value of a projective measurement 
\begin{align*}
    \mathbb{E}[M] = \sumX_m \lambda_m p(\lambda_m) & = \sumX_m \lambda_m \BraKet{P_m} \\ 
    & = \BraKet{\sumX_m \lambda_m P_m} = \BraKet{M} \eqqcolon \braket{M} \hspace{10pt} | \text{if \ketp is clear from context.}
\end{align*}

\noindent
Corresponding \underline{standard deviation}:
\begin{align*} 
    \Delta(M) & \coloneqq \sqrt{\braket{M^2} - \braket{M}^2} \hspace{15pt} | \text{with } \braket{M^2} = \BraKet{M^2} \\
              & = \sqrt{\braket{M - \braket{M}}^2}
\end{align*}

Examples:
\begin{itemize}
    \item Measuring a qubit w.r.t. (with respect to) computational basis \compbas is actually a projective measurement: \(P_0 = \ketz \braz, P_1 = \keto \brao\) \\
            \textcolor{gray}{$p(0) = \BraKet{P_0} = \braket{\psi | 0} \braket{0 | \psi} = |\braket{0 | \psi}|^2$} \\
            \textcolor{gray}{$\ketp = \super \leadsto |\braket{0 | \psi}|^2 = |\alpha|^2$}
    \item In general: measurement w.r.t. orthogonal basis ${\ket{u_1}, \ket{u_2}}$ is a projective measurement:
            \[\text{set } P_m = \ket{u_m} \bra{u_m} \text{ for } m = 1,2\]
            Define observable M by
            \[M \coloneqq \sumX_{m=1}^2 \lambda_m P_m \text{ with arbitrary } \lambda_1, \lambda_2 \in \mathbb{R}, \lambda_1 \ne \lambda_2\]
    \item Measuring Pauli-Z \textcolor{gray}{(as observable)}:
            \[z = 1 \cdot \underbrace{\colvec{1 & 0 \\ 0 & 0}}_{P_0 = \ketz \braz} + (-1) \underbrace{\colvec{0 & 0 \\ 0 & 1}}_{P_1 = \keto \brao} \]
            Agrees with standard measurement w.r.t. the computational basis \compbas.
\end{itemize}

\subsection{The Heisenberg uncertainty principle} 
Supose C and D are two observables and \ketp a quantum state. \\
Then:
\[\Delta(C) \cdot \Delta(D) \ge \frac{|\BraKet{[C, D]}|}{2} \text{ \hspace{15pt} \textcolor{gray}{$[A, B] \coloneqq AB - BA$}}\]
Interpretation for experiment: repeated preparation of \ketp, measure C in some cases and D in the other cases to obtain the standard deviations $\Delta(C)$ and $\Delta(D)$. \\
\textcolor{gray}{
    \begin{align*} 
        \text{Remark: "popular statement: } C \; &\hat{=} \; \hat{x} \text{ (position operator)} \\
         D \; &\hat{=} \; \hat{p} \text{ (momentum operator)}
    \end{align*}
    $\leadsto$ see handout Heisenberg-uncertainty-principle.pdf for derivation.
}